\documentclass{article}
\usepackage{amsmath}
\usepackage{amsfonts}
\usepackage{amssymb}
\usepackage{graphicx}
\usepackage{booktabs}
\usepackage{hyperref}
\usepackage[a4paper, total={6in, 8in}]{geometry}

\newcommand{\nregions}{n^{\text{reg}}}
\newcommand{\nconnections}{n^{\text{conn}}}

\newcommand{\ncore}{n^\text{core}}
\newcommand{\ncopy}{n^\text{copy}}

\newcommand{\stateCore}{x}
\newcommand{\stateCopy}{z}

\newcommand{\pf}{g^{\text{pf}}}
\newcommand{\busspecs}{g^{\text{bus}}}

%opening
\title{Morenet documentation}
\author{Tillmann Mühlpfordt}

\begin{document}

\maketitle

\begin{abstract}
This documentation gathers relevant information pertaining to the modeling and solution of the distributed power flow problem for the morenet project.
\end{abstract}

\section{Local power flow problem}
We compose the set of buses of each region $i$ into a set of \emph{core nodes}, and a set of \emph{copy nodes}.
The core nodes are the original nodes that make up the power system.
The copy nodes are the nodes that connect the power system to the neighboring power systems.
Hence, the copy nodes \emph{do not belong to the power system} but to its neighbors.
The sole purpose of the copy nodes is to store the complex voltage information at the neighboring buses such that power flow equations can be constructed.

The state of the core nodes contains the voltage angles, the voltage magnitudes, the net active power, and the net reactive power of all core nodes
\begin{align}
    \label{eq:state-core}
    \stateCore_i = \begin{bmatrix}
        \theta_i^\text{core} \\
        v_i^\text{core} \\
        p_i^\text{core} \\
        q_i^\text{core} \\
    \end{bmatrix}
    \in \mathbb{R}^{4 \ncore_i}.
\end{align}

The state of the copy nodes contains the voltage angles and the voltage magnitudes of all copy nodes
\begin{align}
    \label{eq:state-copy}
    \stateCopy_i = \begin{bmatrix}
        \theta_i^\text{copy} \\
        v_i^\text{copy}
    \end{bmatrix}
    \in \mathbb{R}^{2 \ncopy_i}.
\end{align}
Hence, from \eqref{eq:state-core} and \eqref{eq:state-copy} we see: each region $i$ is described by a total of $4 \ncore_i + 2 \ncopy_i$ real numbers.

\subsection{Power flow equations}
In region $i$ we formulate a total of $2 \ncore_i$ power flow equations $\pf_i \colon \mathbb{R}^{4 \ncore_i} \times \mathbb{R}^{2 \ncopy_i} \rightarrow \mathbb{R}^{2 \ncore_i}$ for all core nodes
\begin{align}
    \pf_i( \stateCore_i, \stateCopy_i) = 0.
\end{align}

\subsection{Bus specifications}
In region $i$ we formulate a total of $2 \ncore_i$ bus specifications $\busspecs_i \colon \mathbb{R}^{4 \ncore_i} \rightarrow \mathbb{R}^{2 \ncore_i}$ for all core nodes
\begin{align}
    \busspecs_i( \stateCore_i) = 0.
\end{align}

\subsection{Degrees of freedom}

Subtracting the number of equations from the number of decision variables gives us a total of
\begin{align}
    \underbrace{4 \ncore_i + 2 \ncopy_i}_{\text{Decision variables}} - \underbrace{2 \ncore_i}_{\text{Power flow equations}} - \underbrace{2 \ncore_i}_{\text{Bus specifications}} = \underbrace{2 \ncopy_i}_{\text{Degrees of freedom}}
\end{align}
degrees of freedom for each region~$i$.
These degrees of freedom must be fixed globally by the consensus constraints.

\begin{table}
    \centering
    \caption{Symbols and their meanings.\label{tab:symbols-and-meanings}}
    \begin{tabular}{ll}
        \toprule
        Symbol & Meaning \\
        \midrule
        $\nregions$ & Number of regions \\
        $\nconnections$ & Number of connecting lines between regions \\
        $\ncore_{i}$ & Number of core nodes in region $i$ \\
        $\ncopy_{i}$ & Number of copy nodes in region $i$ \\
        \midrule
        $\stateCore_{i}$ & State of core nodes in region $i$ \\
        $\stateCopy_{i}$ & State of copy nodes in region $i$ \\
        \bottomrule
    \end{tabular}
\end{table}

\section{Decentralized power flow problem}

\begin{subequations}
    \begin{align}
        \pf_i( \stateCore_i, \stateCopy_i ) &= 0 \\
        \busspecs_i ( \stateCore_i ) &= 0 \\
        \sum_{i = 1}^{\nregions} A_i \begin{bmatrix}
            \stateCore_i \\
            \stateCopy_i
        \end{bmatrix}
        &= 0,
    \end{align}
\end{subequations}
where the consensus matrices $A_i \in \mathbb{R}^{4 \nconnections \times (4 \ncore_i + 2 \ncopy_i)}$ enforce equality of the voltage angle and the voltage magnitude at the copy buses and their respective original buses.

\end{document}
