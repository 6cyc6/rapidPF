\documentclass{article}
\usepackage{amsmath}
\usepackage{amsfonts}
\usepackage{amssymb}
\usepackage{graphicx}
\usepackage{booktabs}
\usepackage{hyperref}
\usepackage{xcolor}
\hypersetup{
    colorlinks,
    linkcolor={red!50!black},
    citecolor={blue!50!black},
    urlcolor={blue!80!black}
}
\usepackage[a4paper, total={6in, 8in}]{geometry}
\usepackage{libertine}
\usepackage{multirow}

\newcommand{\n}{n}

\newcommand{\nregions}{\n^{\text{reg}}}
\newcommand{\nconnections}{\n^{\text{conn}}}

\newcommand{\ncore}{\n^\text{core}}
\newcommand{\ncopy}{\n^\text{copy}}

\newcommand{\state}{x}
\newcommand{\stateCore}{x}
\newcommand{\stateCopy}{z}

\newcommand{\pf}{g^{\text{pf}}}
\newcommand{\busspecs}{g^{\text{bus}}}

\newcommand{\norm}[1]{\left\lVert#1\right\rVert}

%opening
\title{Morenet documentation}
\author{Tillmann Mühlpfordt}

\begin{document}

\maketitle

\begin{abstract}
This documents sketches the idea to solve the distributed power flow problem as a distributed nonlinear least-squares problem.
\end{abstract}

\section{Problem formulation}
The mathematical formulation for the decentralized power flow problem reads
\begin{subequations}
    \label{eq:dist-power-flow-problem}
    \begin{align}
        \pf_i( \stateCore_i, \stateCopy_i ) &= 0, \\
        \busspecs_i ( \stateCore_i ) &= 0, \\
        \sum_{i = 1}^{\nregions} A_i \begin{bmatrix}
            \stateCore_i \\
            \stateCopy_i
        \end{bmatrix}
        &= 0,
    \end{align}
\end{subequations}
where the consensus matrices $A_i \in \mathbb{R}^{4 \nconnections \times (4 \ncore_i + 2 \ncopy_i)}$ enforce equality of the voltage angle and the voltage magnitude at the copy buses and their respective original buses.
Mathematically speaking, Problem~\ref{eq:dist-power-flow-problem} is a system of nonlinear equations; there are as many equations as there are unknowns.
In principle, Problem~\ref{eq:dist-power-flow-problem} can be solved by Newton's method.
However, we would like to explore alternatives to Newton's method, such as distributed optimization.

\section{Problem solution}
Currently, we solve Problem~\ref{eq:dist-power-flow-problem} as a distributed \emph{feasibility} probem, namely
\begin{subequations}
    \label{eq:dist-feasibility-problem}
    \begin{align}
        \underset{\stateCore_i, \stateCopy_i \, \forall i \in  \{1, \dots, \nregions\}}{\operatorname{min}} \: 0 \quad \operatorname{s.t.}\\
        \pf_i( \stateCore_i, \stateCopy_i ) &= 0, \\
        \busspecs_i ( \stateCore_i ) &= 0, \\
        \sum_{i = 1}^{\nregions} A_i \begin{bmatrix}
            \stateCore_i \\
            \stateCopy_i
        \end{bmatrix}
        &= 0.
    \end{align}
\end{subequations}
From our experience so far, we can say that Aladin is a viable method, but ADMM is not.
Solving a distributed feasibility problem, however, is not the only way.
We can also think of solving Problem~\ref{eq:dist-power-flow-problem} as a distributed least-squares problem of the form
\begin{subequations}
    \label{eq:dist-least-squares-problem}
    \begin{align}
        \underset{\stateCore_i, \stateCopy_i \, \forall i \in  \{1, \dots, \nregions\}}{\operatorname{min}} \: \norm{\begin{bmatrix}
            \pf_i( \stateCore_i, \stateCopy_i ) \\
            \busspecs_i ( \stateCore_i )
        \end{bmatrix}}^2 \quad \operatorname{s.t.} ~ \sum_{i = 1}^{\nregions} A_i \begin{bmatrix}
            \stateCore_i \\
            \stateCopy_i
        \end{bmatrix}
        = 0.
    \end{align}
\end{subequations}
Clearly, the solution to Problem~\ref{eq:dist-feasibility-problem} is the solution to Problem~\ref{eq:dist-least-squares-problem} (why?).
Hence, we can think of solving the least-squares Problem~\ref{eq:dist-least-squares-problem} as a necessary condition.

\subsection{Next steps}

The goal is to explore how to solve the distributed power flow problem via a distributed least-sqaures Problem~\ref{eq:dist-least-squares-problem}.
To do so, there are a couple of immediate next steps
\begin{itemize}
    \item Familiarize with nonlinear least-squares (Gauss-Newton, Levenberg-Marquardt, etc)
    \item Solve a prototypical distributed power flow problem.
    \item Use the Jacobian matrix that is given analytically as a return value from \href{https://iai-vcs.iai.kit.edu/advancedcontrol/code/morenet/morenet/-/blob/master/03_parser/generate_distributed_problem.m}{\texttt{generate$\_$distributed$\_$problem}}.
    \item Work out how the Aladin algorithm simplifies in the absence of both equality and inequality constraints. (Note that inequality constraints in the form of lower/upper bounds might still be necessary, i.e. $\underline{x} \leq x \leq \overline{x}$)
    \item Exploit the problem structure as much as possible. As a general rule, the bigger the problem, the more it pays off to exploit sparsity etc.
\end{itemize}
\autoref{tab:knowledge-base} lists the experience we have gained thus far.
The goal is to fill out everything that is related to least squares.

\begin{table}
    \centering
    \caption{Our experience so far with solving Problem~\ref{eq:dist-power-flow-problem}. A ``---'' indicates further investigations are required.\label{tab:knowledge-base}}
    \begin{tabular}{llllp{4cm}}
        \toprule
        Method & Solver & Number of iterations & Wall clock time & Remark\\
        \midrule
        ADMM & Casadi \& Ipopt & many & not acceptable\\
        \midrule
        \multirow{5}{*}{Aladin} & Casadi \& Ipopt & few & acceptable & does not scale well to larger problems ($N \approx 100$)\\
         & fmincon \& Jacobian & few & acceptable & tends to become slow for larger problems ($N \approx 300$)\\
         & Ipopt \& Jacobian & --- & --- & ---\\
         \midrule
        Least squares & --- & --- & --- & ---\\
        \bottomrule
    \end{tabular}
\end{table}

\end{document}
